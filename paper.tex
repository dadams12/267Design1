% TEMPLATE for Usenix papers, specifically to meet requirements of
%  USENIX '05
% originally a template for producing IEEE-format articles using LaTeX.
%   written by Matthew Ward, CS Department, Worcester Polytechnic Institute.
% adapted by David Beazley for his excellent SWIG paper in Proceedings,
%   Tcl 96
% turned into a smartass generic template by De Clarke, with thanks to
%   both the above pioneers
% use at your own risk.  Complaints to /dev/null.
% make it two column with no page numbering, default is 10 point

% Munged by Fred Douglis <douglis@research.att.com> 10/97 to separate
% the .sty file from the LaTeX source template, so that people can
% more easily include the .sty file into an existing document.  Also
% changed to more closely follow the style guidelines as represented
% by the Word sample file. 

% Note that since 2010, USENIX does not require endnotes. If you want
% foot of page notes, don't include the endnotes package in the 
% usepackage command, below.

% This version uses the latex2e styles, not the very ancient 2.09 stuff.
\documentclass[letterpaper,12pt]{article}
\usepackage{epsfig,graphicx,usenix,fullpage, hyperref}


%\usepackage{endnotes}
\begin{document}


%make title bold and 14 pt font (Latex default is non-bold, 16 pt)
\title{\Large \bf 6.267 Design Project 1}
%for single author (just remove % characters)
\author{
{\rm Colleen Josephson}\\
cjoseph@mit.edu
\and
{\rm Dave Adams}\\
dcadams@mit.edu
% copy the following lines to add more authors
% \and
% {\rm Name}\\
%Name Institution
} % end author

\date{Nov. 8, 2013}

\maketitle

% Use the following at camera-ready time to suppress page numbers.
% Comment it out when you first submit the paper for review.
%\thispagestyle{empty}

\section{Introduction}

Wireless networks introduce many difficult challenges, and the increasing 
mobilty of users only exacerbates these issues. Users want to be able to walk 
from place to place with continuous and automatic network connectivity. Most
modern wireless networks have quite poor support for moving users, however.
The connection must usually be re-established when the user finally becomes
stationary. The problem is that by the time a moving user establishes a 
connection with an access point, they are moving out of range. 

Wireless access points have a relatively stable topology in most buildings.
This information can be used to figure out where a user is likely to go 
next so the access point can be ready to interact with the user seamlessly.
Most currently deployed schemes do not take advantage of these properties.
We aim 

\subsection{Assumptions}

Throughout the design process, we worked with few important assumptions, which
we believe are reasonable. These are outlined below.

\begin{itemize}
\item The wired network is much faster than the wireless network
\item Movement of the devices will typically be no faster than 5mph (a slow jog)
\item Computation is cheap at the gateway, slightly less cheap at hotspots.
\item All hotspots are one hop away from the gateway
\item The access points and mobile users are capable of MIMO. Users have at 
least two antennas. 
\item There will be a pre-computed table of the nearest neighbors to any 
hotspot, developed from floorplans and kept updated by the network manager
\end{itemize}

\section{Design} \label{sec:design}

In this section we discuss our design goals, and outline how our system
meets these goals. We discuss the details of the routing protocol and provide 
examples of typical operating scenarios.

\subsection{Goals}
The primary goals of this system is to minimize the amount of user 
intervention that is necessary, and to maximize the connectivity.
We aim to never have the user unable to send or receive packets while 
transitioning from access point to access point, unless the user enters an area
that has no coverage, such as a tunnel or elevator. If a user does lose coverage,
we aim to have the user rapidly reconnected as soon as they are in range again.
We do not to waste too much capacity ensuring these smooth coverage area 
ransitions--our goals above could be met by simply broadcasting the packets
from every access point, but this would lead to terrible throughput.

\subsection{Protocol Outline}

Scenarios: user walking down hall, stationary users, users walking past
a heavily loaded access point (e.g. lecture hall).

\begin{itemize}
\item Use floor plans to create a list of 2 nearest neighbors to any access point

\item use nearest neighbor list as basis for a markov chain

\item markov chain is further developed by watching patterns of where users tend 
to move to from a given state (this is particularly helpful in the case that a user
enters an elevator or tunnel, since movement is not continuous there)

\item the router broadcasts to a subset of the nodes, which includes the node 
that the user's SNR is strongest at as well as at least 2 neighboring nodes

\item how to select how many nodes to transmit to, if many are possible in the
list of next states? Thought: select nodes whose probability of being the next 
state is at least 20 percent, or 3 nodes (whichever is greater). If no nodes
have a prbability of at least 20 percent, as may be the case with a node next 
to an elevator that goes up to 10 equally likely floors, then we ignore the 20 
percent rule and broadcast to some alternative subset the network maintainer sets.
Continuing the elevator example, if the tallest building in the network contains
ten stories, then a reasonable alternative would be broadcasting to all ten
access points in the elevator lobbies.

Alternatively, different classes of access point could exist and be input
permanetly into the gateway. An elevator lobby access point would have one
set of rules, the hallway access points another, etc. This would be 
much more annoying to set up, but it would only have to be done once.

\item each access point periodcially takes a snapshot of all the users it has
talked to in the past period and the average SNR, and sends this information
to the gateway. The gateway then makes decisions on which access points
should broadcast for which users, and sends out that information.
A user is considered to  have changed states if the SNR is now stronger at a 
different access point than before. 

\item upon receiving updates, gateway also updates markov model before deleting
previous state information. This way, we have a learning markov model!

\item have users send out chirps at least once per second so the APs can 
maintain accurate livliness and SNR data for users who an're xmitting 
continuously? Don't think it will add much overhead, since we can piggyback 
it onto something like the wireless rate selection probe packets.

\end{itemize}

\subsubsection{Markov Model}
How is the markov model maintained? Talk about markov maintenance overhead
in performance section.

\section{Analysis}

In this section we analyze how well our design performs. We also discuss 
engineering tradeoffs, and possible solutions to new problems introduced by 
our approach.

\subsection{Performance}
One primary concern is the amount of routing traffic necessary to track
users satisfactoraly. We have designed the system to perform reliably
up to movements of 5pmh, which is a slow jog. 5mph translates to approximately
2.24 meters per second. If we assume that access points are typically 10 or more
meters apart, then a user will be reasonably tracked if the access points
send one update packet per second. If moving continuously down a well-covered
hallway, a user moving less than 5mph will take 4 or more seconds to pass
through an access point's coverage area. 

How reasonable is the traffic incurred by one routing update every 2 seconds? 
If the gateway responds to each update period before the next period starts,
this means an average of 1 update per wired link per second (one for the 
access point, and another for the gateway's response). Are we going to do
any TCP/reliability, or assume that the wire line is reliable enough?
Are the links bi-directional, or will traffic to the gateway have to
compete with traffic from the gateway?

Another concern is interference caused by multiple access points broadcasting 
the same packets simultaneously. This leads to a reduction in the number
of possible users in the system, since each user has at least three times
as much traffic. Additionally, the spatial overlap of some access points leads
to concerns about destructive interference. 

A performance hit is unavoidable if each user is suddenly causing more packets
to be broadcast across the system. If an average  user has 3 access points
broadcasting to it simultaneously, then the system can support only a third of
the users it did before. A performance cut of small constant factor is quite
reasonable, however. It can be somewhat ameliorated by simply adding more access 
points. We say somewhat because suddenly increasing density of the network could
have unexpected consequences, such as the destructive interference concern 
mentioned above. In most systems, additional access points will probably only
need to be added in areas that tend to get very crowded, such as a lecture hall,
since many wireless networks often have mostly lightly loaded access points.

One possible way to reduce interference concerns is to use a network coding 
scheme so that the acess points are not sending identical packets. However,
routing and tracking of users is the primary purpose of this protocol so 
we will only suggest this idea as a possible extension.

\subsection{Edge Cases}
Exceeding 5mph, transience through areas with no hotspots, gateway failure, 
hotspot failure, sudden device disconnects, very high and very low loads.

\section{Conclusion}
blah blah blah

{\footnotesize \bibliographystyle{acm}
\bibliography{paper}}

\end{document}






