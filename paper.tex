% TEMPLATE for Usenix papers, specifically to meet requirements of
%  USENIX '05
% originally a template for producing IEEE-format articles using LaTeX.
%   written by Matthew Ward, CS Department, Worcester Polytechnic Institute.
% adapted by David Beazley for his excellent SWIG paper in Proceedings,
%   Tcl 96
% turned into a smartass generic template by De Clarke, with thanks to
%   both the above pioneers
% use at your own risk.  Complaints to /dev/null.
% make it two column with no page numbering, default is 10 point

% Munged by Fred Douglis <douglis@research.att.com> 10/97 to separate
% the .sty file from the LaTeX source template, so that people can
% more easily include the .sty file into an existing document.  Also
% changed to more closely follow the style guidelines as represented
% by the Word sample file. 

% Note that since 2010, USENIX does not require endnotes. If you want
% foot of page notes, don't include the endnotes package in the 
% usepackage command, below.

% This version uses the latex2e styles, not the very ancient 2.09 stuff.
\documentclass[letterpaper,10pt]{article}
\usepackage{epsfig,graphicx,usenix,fullpage, hyperref}


%\usepackage{endnotes}
\begin{document}


%make title bold and 14 pt font (Latex default is non-bold, 16 pt)
\title{\Large \bf 6.267 Design Project 1}
%for single author (just remove % characters)
\author{
{\rm Colleen Josephson}\\
cjoseph@mit.edu
\and
{\rm Dave Adams}\\
dcadams@mit.edu
% copy the following lines to add more authors
% \and
% {\rm Name}\\
%Name Institution
} % end author

\date{Nov. 11, 2013}

\maketitle

% Use the following at camera-ready time to suppress page numbers.
% Comment it out when you first submit the paper for review.
%\thispagestyle{empty}

\section{Introduction}
blah blah blah

\subsection{Assumptions}

Throughout the design process, we worked with few important assumptions, which
we believe are reasonable. These are outlined below.

\begin{itemize}
\item The wired network is much faster than the wireless network
\item Movement of the devices will typically be no faster than 5mph (a slow jog)
\item Computation is cheap at the gateway, slightly less cheap at hotspots.
\item All hotspots are one hop away from the gateway
\item The access points and mobile users are capable of MIMO. Users have at 
least two antennas. 
\item There will be a pre-computed table of the nearest neighbors to any 
hotspot, developed from floorplans and kept updated by the network manager
\end{itemize}

\section{Design} \label{sec:design}

In this section we discuss our design goals, and outline how our system
meets these goals. We discuss the details of the routing protocol and provide 
examples of typical operating scenarios.

\subsection{Goals}
The primary goals of this system is to minimize the amount of user 
intervention that is necessary, and to maximize the connectivity.
We aim to never have the user unable to send or receive packets while 
transitioning from access point to access point, unless the user enters an area
that has no coverage, such as a tunnel or elevator. If a user does lose coverage,
we aim to have the user rapidly reconnected as soon as they are in range again.
We do not to waste too much capacity ensuring these smooth coverage area 
ransitions--our goals above could be met by simply broadcasting the packets
from every access point, but this would lead to terrible throughput.

\subsection{Protocol Outline}

Scenarios: user walking down hall, stationary users, users walking past
a heavily loaded access point (e.g. lecture hall).

\section{Analysis}

In this section we analyze how well our design performs. We also discuss 
engineering tradeoffs, and possible solutions to new problems introduced by 
our approach.

\subsection{Performance}
How often do the access points need to send updates in order for the
user to never lose coverage if walking continuously in a complete coverage 
area at 5mph or less? Is this load reasonable?

\subsection{Edge Cases}
Exceeding 5mph, transience through areas with no hotspots, gateway failure, 
hotspot failure, sudden device disconnects, very high and very low loads.

\section{Conclusion}
blah blah blah

{\footnotesize \bibliographystyle{acm}
\bibliography{paper}}

\end{document}






